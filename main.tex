\documentclass[10pt, a4paper]{article}
\usepackage[czech]{babel}
\usepackage[T1]{fontenc}
\usepackage[utf8]{inputenc}
\usepackage{enumitem}
\usepackage{parskip}
\usepackage{tocloft}
\usepackage{multicol}
\usepackage[hidelinks]{hyperref}
\usepackage{graphicx}
\usepackage{float}
\usepackage{mathtools}
\usepackage{array}
\usepackage{tabularx}
\usepackage{cite}
\usepackage{listings}

\begin{document}

% =====================================
% Titulní strana
% =====================================
\begin{titlepage}
    \includegraphics[width=0.80\textwidth]{images/fav.pdf}
    \begin{center}
        \vspace{2cm}

        \LARGE
        Semestrální práce z KIV/PRO

        \Huge
        \textbf{Časově efektivní algoritmus A* pro plánování trasy robota}

        \vspace{1cm}

        \vfill

        \vspace{0.5cm}

        \normalsize
        \raggedright
        Autor: Vít Novotný \hfill Datum: \today

        \vspace{0.2cm}

    \end{center}
\end{titlepage}

\thispagestyle{empty}
\pagebreak

% =====================================
% Obsah
% =====================================
\setcounter{page}{2}
\renewcommand{\cftsecleader}{\cftdotfill{\cftdotsep}}
\renewcommand{\cftsubsecleader}{\cftdotfill{\cftdotsep}}
\renewcommand{\cftsubsubsecleader}{\cftdotfill{\cftdotsep}}

\tableofcontents
\pagebreak

% =====================================
% Tělo dokumentu
% =====================================

\section{Úvod}
Tato práce vychází ze článku \textit{Time-Efficient A* Algorithm for Robot Path Planning}\cite{MainWork}, která pojednává o problematice efektivního využití algoritmu A* pro plánování trasy robotů v průmyslu. Tento dokument je psán tak, aby důležité informace byly pochopitelné pro studenta 2. ročníku informatiky.



\section{Vývoj a použité technologie}
[Popiš, jaké technologie nebo nástroje byly použity.]

\section{Implementace řešení}

\section{Ovládání programu}
[Vysvětli, jak se program používá, jaké má vstupy a výstupy.]

\section{Definované proměnné}
[Vlož tabulku s proměnnými, pokud je potřeba.]

\begin{center}
    \begin{tabularx}{0.8\textwidth}{
        | >{\raggedright\arraybackslash}X
        | >{\raggedright\arraybackslash}X
        | >{\raggedright\arraybackslash}X | }
         \hline
         Název & Paměťová adresa & Velikost [bajty] \\
         \hline
         [doplň údaje] &  &  \\
         \hline
    \end{tabularx}
\end{center}

\section{Závěr}
[Zhodnoť výsledky své práce, popiš, co ses naučil a s jakými problémy ses potýkal.]

\newpage
\bibliographystyle{plain}
\bibliography{citations}


\end{document}
