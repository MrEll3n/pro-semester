\documentclass[10pt, a4paper]{article}
\usepackage[czech]{babel}
\usepackage[T1]{fontenc}
\usepackage[utf8]{inputenc}
\usepackage{enumitem}
\usepackage{parskip}
\usepackage{tocloft}
\usepackage{multicol}
\usepackage[hidelinks]{hyperref}
\usepackage{graphicx}
\usepackage{float}
\usepackage{mathtools}
\usepackage{array}
\usepackage{tabularx}
\usepackage{cite}
\usepackage{listings}

\begin{document}

% =====================================
% Titulní strana
% =====================================
\begin{titlepage}
    \includegraphics[width=0.80\textwidth]{images/fav.pdf}
    \begin{center}
        \vspace{2cm}

        \LARGE
        Semestrální práce z KIV/PRO

        \Huge
        \textbf{Časově efektivní algoritmus A* pro plánování trasy robota}

        \vspace{1cm}

        \vfill

        \vspace{0.5cm}

        \normalsize
        \raggedright
        Autor: Vít Novotný \hfill Datum: \today

        \vspace{0.2cm}

    \end{center}
\end{titlepage}

\thispagestyle{empty}
\pagebreak

% =====================================
% Obsah
% =====================================
\setcounter{page}{2}
\renewcommand{\cftsecleader}{\cftdotfill{\cftdotsep}}
\renewcommand{\cftsubsecleader}{\cftdotfill{\cftdotsep}}
\renewcommand{\cftsubsubsecleader}{\cftdotfill{\cftdotsep}}

\tableofcontents
\pagebreak

% =====================================
% Tělo dokumentu
% =====================================

\section{Úvod}
Tato práce vychází ze článku \textit{Time-Efficient A* Algorithm for Robot Path Planning}\cite{MainWork}, která pojednává o problematice efektivního využití algoritmu A* pro plánování trasy robotů v průmyslu. Všechna data, tabulky a obrázky jsou využity z již zmíněného vycházející práce. Tento dokument je psán tak, aby důležité informace byly pochopitelné pro studenta 2. ročníku informatiky.

Plánování trasy (angl. path planning) představuje jeden ze základních problému v oblasti robotiky. Cílem je nalést optimální cestu bez kolizí mezi počátečním a cílovým bodem v daném prostředí. Lze rozlišit dva hlavní přístupy plánování – \textbf{Offline} a \textbf{Online}

Při \textbf{Offline} předpokládáme, že prostředí je předem známo a všechny překážky jsou nehybné, což umožňuje předání kompletní geometrie pracovního prostředí jakožto vstup algoritmu.

Oproti tomu \textbf{Online} přístup využívá data ze senzorů v reálném čase a trajektorii vypočítává za běhu podle aktuální situace.

Tento algoritmus se zaměřuje na offline plánování trasy v prostředí s pevně určenými překážkami.

Mobilní roboti (autonomní robot, který za použití trasovacích algoritmů hledá optimální cestu) mohou využít různé typy algoritmů, jako např. genetické algoritmy, umělé neuronové sítě (ANN) či A* algoritmus, který bývá užíván nejčastěji. Kombinuje heuristický odhad s hodnotou náladů jednotlivých uzlů v mřížce, čímž umožnuje nalést řešení blízké optimu.

Klasický A* algoritmus s sebou nese určité nevýhody - při větším počtu uzlů je výpočet časově náročný a jeho výkon výrazně klesá zejména při využití procesorů s nižším výpočetním výkonem. Autoři článku navrhují vylepšenou verzi A* algorimtu, která tyto problémy částečně eliminuje.

Nový přístup počítá heuristickou funkci pouze těsně před kolizí, čímž se výrazně snižuje počet zbytečných výpočtů. Díky tomu lze algoritmus efektivně používat i na starších nebo méně vykonných zařízeních, aniž by ztratil klíčové vlastnosti původního A*.
Při velkých vzdálenostech mezi počátečním a cíloým bodem lze dosáhnout zkrácení doby výpočtu až o 75 \% při minimálním prodloužení cesty.


\section{Navrhovaný algoritmus a jeho podmínky}
Klasický A* algoritmus určuje nejkratší cestu mezi dvěma body v grafu (v našem případě v mřížce) na základě součtu dvou hodnot: \textbf{g(x)} a \textbf{h(x)}.

Funkce \textbf{g(x)} představuje skutečné náklady cesty od počátečního bodu k aktuálnímu uzlu, zatímco \textbf{h(x)} je heuristický odhad vzdálenosti od aktuálního uzlu k cíli.\cite{WikiAStar}.
Celkovou hodnotu každého uzlu určuje vztah $f(x) = g(x) + h(x)$.
Uzel $f(x)$ využijeme jako nový aktuální.

Tento přístup zaručuje nalezení optimální nebo téměř optimální cesty, ale může být značně výpočetně náročný, zejména pokud je prostředí rozsáhlé nebo obsahuje mnoho překážek.

Vylepšený algoritmus, označovaný jako Time-Efficient A* (časově efektivní A*) algoritmus, se snaží tento problém řešit a snížit výpočetní náročnost.

Hlavní rozdíl spočívá v tom, že funkce $g(x)$ je počítána pouze před detekcí kolize, zatímco kompletní $f(x)$ je počátán až těsně před a po kolizním bodě.

Tím se eliminuje nutnost neustálého výpočtu heuristiky ve všech uzlech, což významě redukuje celkový výpočetní čas.
Díky tomuto přístupu je robot schopen rychleji vykonávat své úlohy, jelikož algoritmus počítá samotný A* až v bodě, kdy je opravdu potřeba.

Celkový postup Time-Efficient A* je složen z pěti fází:

\subsection{Tvorba vstupních dat}
\begin{itemize}
    \item

\end{itemize}

\subsection{Vzdálenostní kontrola}

\subsection{Výpočet náklonu a předpočet trasy}

\subsection{Detekce kolize s překážkou}

\subsection{Přechod na algoritmus A*}

\section{Definované proměnné}
[Vlož tabulku s proměnnými, pokud je potřeba.]

\begin{center}
    \begin{tabularx}{0.8\textwidth}{
        | >{\raggedright\arraybackslash}X
        | >{\raggedright\arraybackslash}X
        | >{\raggedright\arraybackslash}X | }
         \hline
         Název & Paměťová adresa & Velikost [bajty] \\
         \hline
         [doplň údaje] &  &  \\
         \hline
    \end{tabularx}
\end{center}

\section{Závěr}
[Zhodnoť výsledky své práce, popiš, co ses naučil a s jakými problémy ses potýkal.]

\newpage
\bibliographystyle{plain}
\bibliography{citations}


\end{document}
